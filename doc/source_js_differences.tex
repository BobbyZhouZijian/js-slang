\section*{Deviations from JavaScript}

We intend the 
Source language to be a conservative extension
of JavaScript: Every correct
Source program should behave \emph{exactly} the same
using a Source implementation, as it does using a JavaScript
implementation. We assume, of course, that suitable libraries are
used by the JavaScript implementation, to account for the predefined names
of each Source language.

This section lists some exceptions where we think a Source implementation
should be allowed to deviate from the JavaScript specification, for the
sake of internal consistency and esthetics.

\begin{description}
\item[Empty block as last statement of toplevel sequence:] In JavaScript,
  empty blocks as last statement of a sequence are apparently
  ignored. Thus the result of evaluating such a sequence is the
  result of evaluating the previous statement. Implementations
  of Source might stick to the more intuitive result: \texttt{undefined}.
  Example:
  \begin{lstlisting}
1;
{
  // empty block
}
  \end{lstlisting}
  The result of evaluating this program can be \texttt{undefined}
  for implementations of Source. Note that this issue only arises
  at the toplevel---outside of functions.
\item[Literal object notation at toplevel:] JavaScript implementations support
  the creation of literal objects using \verb#{...}#. Occasionally,
  ambiguities with block syntax arises as a result, compounded by
  optional semicolons in the JavaScript syntax. Apparently,
  literal objects are preferred in JavaScript in some such cases.
  Implementations
  of Source might interpret some programs as blocks, instead. Example:
  \begin{lstlisting}
{
  // empty block
}    
  \end{lstlisting}
  The result of evaluating this program can be \texttt{undefined}
  for implementations of Source.
  Note that this issue only arises
  at the toplevel---outside of functions.  
\end{description}
